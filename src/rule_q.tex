\section*{Rule~Q: Preemptive transition firing}\label{sec:rule_q}
Rule~Q evaluates transitions that are initially enabled and are the only consumer of all places in its pre set.
The formal description of Rule~Q can be found in Figure~\ref{fig:rule_q}.
Remark that Rule~Q can potentially put tokens into places which will prevent other reductions.
Furthermore, it can be applied infinitely if $\boxminus(t_0)\leq \boxplus(t_0)$, or if the Petri net contains a loop.

\begin{figure}[h!]
    \centering
    \begin{tikzpicture}
        % Left side places
        \node[place] (lplace1) at (0,2) {$k_1$};
        \node[place] (lplace2) at (2,2) {$k_n$};
        \node at (1,2) {$\cdots$};
        \node[place] (lplace3) at (0,-2) {$l_1$};
        \node[place] (lplace4) at (2,-2) {$l_m$};
        \node at (1,-2) {$\cdots$};

        % Left side transition
        \node[transition,minimum height=6mm,minimum width=2mm,fill=black,label=left:$t_0$] (ltrans) at (1,0) {};
        %\node[transition,minimum height=6mm,minimum width=2mm,fill=black,label=below:$t_0$] (remTrans1) at (0,-2) {};

        % Left side invisible nodes
        \node (lp1in1) at (-0.5,3.5) {};
        \node (lp1in2) at (0.5,3.5) {};
        \node (lp2in1) at (1.5,3.5) {};
        \node (lp2in2) at (2.5,3.5) {};
        \node (lp3in1) at (-0.5,-0.5) {};
        \node (lp4in1) at (2.5,-0.5) {};
        \node (lp3out1) at (-0.5,-3.5) {};
        \node (lp4out1) at (2.5,-3.5) {};

        % Left side arcs between transitions and nodes
        \draw[-latex,thick] (lplace1) edge node[left] {$w_1$} (ltrans);
        \draw[-latex,thick] (lplace2) edge node[right] {$w_n$} (ltrans);
        \draw[-latex,thick] (ltrans) edge node[left] {$v_1$} (lplace3);
        \draw[-latex,thick] (ltrans) edge node[right] {$v_m$} (lplace4);

        % Left side arcs to/from invisible nodes
        \draw[-latex,thick] (lp1in1) -- (lplace1);
        \draw[-latex,thick] (lp1in2) -- (lplace1);
        \draw[-latex,thick] (lp2in1) -- (lplace2);
        \draw[-latex,thick] (lp2in2) -- (lplace2);
        \draw[-latex,thick] (lp3in1) -- (lplace3);
        \draw[-latex,thick] (lp4in1) -- (lplace4);
        \draw[-latex,thick] (lplace3) -- (lp3out1);
        \draw[-latex,thick] (lplace4) -- (lp4out1);

        % ================== Middle arrow ==================
        \node (arrow) at (4,0) {\huge$\Rightarrow$};
        \node[text width=3.5cm] at (4.9, -1.2) {\textbf{where}\\$\forall i\;.\;k_i\geq w_i$};
        % ==================================================

        % Right side places
        \node[place,minimum size=1.3cm] (rplace1) at (6,2) {$k_1 - w_1$};
        \node at (7,2) {$\cdots$};
        \node[place,minimum size=1.3cm] (rplace2) at (8,2) {$k_n - w_n$};
        \node[place,minimum size=1.3cm] (rplace3) at (6,-2) {$l_1 + v_1$};
        \node at (7,-2) {$\cdots$};
        \node[place,minimum size=1.3cm] (rplace4) at (8,-2) {$l_m + v_m$};

        % Right side transition
        \node[transition,minimum height=6mm,minimum width=2mm,fill=black,label=left:$t_0$] (rtrans) at (7,0) {};
        %\node[transition,minimum height=6mm,minimum width=2mm,fill=black,label=below:$t_0$] (remTrans1) at (0,-2) {};

        % Right side invisible nodes
        \node (rp0in1) at (5.5,3.5) {};
        \node (rp0in2) at (6.5,3.5) {};
        \node (rp1in1) at (7.5,3.5) {};
        \node (rp1in2) at (8.5,3.5) {};
        \node (rp2in1) at (5.5,-0.5) {};
        \node (rp3in1) at (8.5,-0.5) {};
        \node (rp2out1) at (5.5,-3.5) {};
        \node (rp3out1) at (8.5,-3.5) {};

        % Right side arcs between transitions and nodes
        \draw[-latex,thick] (rplace1) edge node[left] {$w_1$} (rtrans);
        \draw[-latex,thick] (rplace2) edge node[right] {$w_n$} (rtrans);
        \draw[-latex,thick] (rtrans) edge node[left] {$v_1$} (rplace3);
        \draw[-latex,thick] (rtrans) edge node[right] {$v_m$} (rplace4);

        % Right side arcs to/from invisible nodes
        \draw[-latex,thick] (rp0in1) -- (rplace1);
        \draw[-latex,thick] (rp0in2) -- (rplace1);
        \draw[-latex,thick] (rp1in1) -- (rplace2);
        \draw[-latex,thick] (rp1in2) -- (rplace2);
        \draw[-latex,thick] (rp2in1) -- (rplace3);
        \draw[-latex,thick] (rp3in1) -- (rplace4);
        \draw[-latex,thick] (rplace3) -- (rp2out1);
        \draw[-latex,thick] (rplace4) -- (rp3out1);
    \end{tikzpicture}
    \vspace{1cm}

    \begin{adjustbox}{center}
        \begin{tabular}{|p{65mm}|p{45mm}|} \hline
        Precondition & Update \\ \hline
        Fix transition $t_0$ s.t.:
        \begin{itemize}[leftmargin=10mm]
            \item[Q1)] $({}^\bullet t)^\bullet = \{t_0\}$
            \item[Q2)] $\boxminus(t_0) \leq M_0 < I(t_0)$
            \item[Q3)] $({}^\bullet t_0 \cup t_0^\bullet) \cap
            places(\varphi) = \emptyset$
            \item[Q4)] $({}^\bullet t_0)^\circ = (t_0^\bullet)^\circ = \emptyset$
        \end{itemize}
        &
        \begin{itemize}[leftmargin=10mm]
            \item[UQ1)] $M_0:=M_0 + E(t_0)$.
        \end{itemize} \\ \hline
        \end{tabular}
    \end{adjustbox}
    \caption{Rule Q: Preemptive transition firing}
    \label{fig:rule_q}
\end{figure}

\begin{theorem}
    Rule~Q in Figure~\ref{fig:rule_q} is correct for CTL\textbackslash X.
\end{theorem}
