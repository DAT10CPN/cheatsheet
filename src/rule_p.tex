\section*{Rule P: Redundant inhibitor arc}\label{sec:rule_p}
Sometimes we can find an upper bound on the number of tokens at a place $p_0$.
This upper bound is given by the initial marking if all transitions have a non-positive effect on $p_0$.
Any inhibitor arc from $p_0$ with a weight higher than the upper bound of $p_0$ therefore never inhibits,
which means the inhibitor arc can be removed.
See Figure~\ref{fig:rule_p} for a formal description of Rule~P.

\begin{figure}[h!]
    \centering
    \begin{tikzpicture}
        % Left side places
        \node[place,label=right:$p_0$] (place1) at (0,0) {};

        % Left side transition
        \node[transition,minimum height=6mm,minimum width=2mm,fill=black,label=above:$t$] (negTrans1) at (0,2) {};
        \node[transition,minimum height=6mm,minimum width=2mm,fill=black,label=below:$t_0$] (remTrans1) at (0,-2) {};

        % Left side invisible nodes
        \node (negIn1) at (-1,3) {};
        \node (negOut1) at (1,3) {};
        \node (placeIn1) at (-1.5,0) {};
        \node (remIn1) at (-1,-3) {};
        \node (remOut1) at (1,-3) {};

        % Left side arcs between transitions and nodes
        \draw[-latex,thick] (negTrans1) edge[bend right] node[left] {$w_2$} (place1);
        \draw[-latex,thick] (place1) edge[bend right] node[right] {$w_1$} (negTrans1);
        \draw[-{Circle[open]},thick] (place1) -- node[left] {$w_0$} (remTrans1);

        % Left side arcs to/from invisible nodes
        \draw[-latex,thick] (negIn1) -- (negTrans1);
        \draw[-latex,thick] (negTrans1) -- (negOut1);
        \draw[-latex,thick] (place1) -- (placeIn1);
        \draw[-latex,thick] (remIn1) -- (remTrans1);
        \draw[-latex,thick] (remTrans1) -- (remOut1);

        % ================== Middle arrow ==================
        \node (arrow) at (3,0) {\huge$\Rightarrow$};
        \node[text width=3.5cm] at (3.5, -2) {\textbf{where}\\$w_0>M_0(p)$\\$w_1\geq w_2$};
        % ==================================================

        % Right side places
        \node[place,label=right:$p_0$] (place2) at (6,0) {};

        % Right side transitions
        \node[transition,minimum height=6mm,minimum width=2mm,fill=black,label=above:$t$] (rTrans2) at (6,2) {};
        \node[transition,minimum height=6mm,minimum width=2mm,fill=black,label=below:$t_0$] (rTrans1) at (6,-2) {};

        % Right side invisible nodes
        \node (invinrt2) at (5,3) {};
        \node (invoutrt2) at (7,3) {};
        \node (placeIn2) at (4.5,0) {};
        \node (invinrt1) at (5,-3) {};
        \node (invoutrt1) at (7,-3) {};

        % Right side arcs between places and transition
        \draw[-latex,thick] (rTrans2) edge[bend right] node[left] {$w_2$} (place2);
        \draw[-latex,thick] (place2) edge[bend right] node[right] {$w_1$} (rTrans2);

        % Right side arcs to/from invisible nodes
        \draw[-latex,thick] (invinrt2) -- (rTrans2);
        \draw[-latex,thick] (rTrans2) -- (invoutrt2);
        \draw[-latex,thick] (place2) -- (placeIn2);
        \draw[-latex,thick] (invinrt1) -- (rTrans1);
        \draw[-latex,thick] (rTrans1) -- (invoutrt1);

    \end{tikzpicture}
    \vspace{1cm}

    \begin{adjustbox}{center}
        \begin{tabular}{|p{65mm}|p{45mm}|} \hline
        Precondition & Update \\ \hline
        Fix place $p_0$ and transition $t_0$ s.t.:
        \begin{itemize}[leftmargin=10mm]
            \item[P1)] $t_0\in p_0^\circ$
            \item[P2)] $I(p_0,t_0)> M_0(p_0)$
            \item[P3)] $^\boxplus p_0 = \emptyset$
        \end{itemize}
        &
        \begin{itemize}[leftmargin=10mm]
            \item[UP1)] $I(p_0,t_0) = \infty$.
        \end{itemize} \\ \hline
        \end{tabular}
    \end{adjustbox}
    \caption{Rule P: Redundant inhibitor arc}
    \label{fig:rule_p}
\end{figure}

\begin{theorem}
    Rule~P in Figure~\ref{fig:rule_p} is correct for CTL*.
\end{theorem}
