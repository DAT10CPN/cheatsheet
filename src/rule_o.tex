\section*{Rule O: Inhibited transition}\label{sec:rule_o}
We can find the lower bound of tokens at a place $p_0$.
Any inhibitor arc from $p_0$ with a weight smaller than the lower bound
always inhibits the given transition, which means that the transition can be removed.
See Figure~\ref{fig:rule_o} for a formal description of Rule~O.

\begin{figure}[h!]
    \centering
    \begin{tikzpicture}
        % Left side places
        \node[place,label=right:$p_0$] (place1) at (0,0) {$\geq low$};

        % Left side transition
        \node[transition,minimum height=6mm,minimum width=2mm,fill=black,label=above:$t$] (negTrans1) at (0,2) {};
        \node[transition,minimum height=6mm,minimum width=2mm,fill=black,label=below:$t_0$] (remTrans1) at (0,-2) {};

        % Left side invisible nodes
        \node (negIn1) at (-1,3) {};
        \node (negOut1) at (1,3) {};
        \node (placeIn1) at (-1.5,0) {};
        \node (remIn1) at (-1,-3) {};
        \node (remOut1) at (1,-3) {};

        % Left side arcs between transitions and nodes
        \draw[-latex,thick] (negTrans1) edge[bend right] node[left] {$w_2$} (place1);
        \draw[-latex,thick] (place1) edge[bend right] node[right] {$w_1$} (negTrans1);
        \draw[-{Circle[open]},thick] (place1) -- node[left] {$w_0$} (remTrans1);

        % Left side arcs to/from invisible nodes
        \draw[-latex,thick] (negIn1) -- (negTrans1);
        \draw[-latex,thick] (negTrans1) -- (negOut1);
        \draw[-latex,thick] (placeIn1) -- (place1);
        \draw[-latex,thick] (remIn1) -- (remTrans1);
        \draw[-latex,thick] (remTrans1) -- (remOut1);

        % ================== Middle arrow ==================
        \node (arrow) at (3,0) {\huge$\Rightarrow$};
        \node[text width=3.5cm] at (3.6, -2) {\textbf{where}\\$w_2>w_1\geq low\geq w_0$};
        % ==================================================

        % Right side places
        \node[place,label=right:$p_0$] (place2) at (6,0) {$\geq low$};

        % Right side transitions
        \node[transition,minimum height=6mm,minimum width=2mm,fill=black,label=above:$t$] (negTrans2) at (6,2) {};

        % Right side invisible nodes
        \node (negIn2) at (5,3) {};
        \node (negOut2) at (7,3) {};
        \node (placeIn2) at (4.5,0) {};

        % Right side arcs between places and transition
        \draw[-latex,thick] (negTrans2) edge[bend right] node[left] {$w_2$} (place2);
        \draw[-latex,thick] (place2) edge[bend right] node[right] {$w_1$} (negTrans2);

        % Right side arcs to/from invisible nodes
        \draw[-latex,thick] (negIn2) -- (negTrans2);
        \draw[-latex,thick] (negTrans2) -- (negOut2);
        \draw[-latex,thick] (placeIn2) -- (place2);

    \end{tikzpicture}
    \vspace{1cm}

    \begin{adjustbox}{center}
        \begin{tabular}{|p{80mm}|p{45mm}|} \hline
        Precondition & Update \\ \hline
        Fix place $p_0$ and transition $t_0$ s.t.:
        \begin{itemize}[leftmargin=10mm]
            \item[O1)] $t_0\in p_0^\circ$
            \item[O2)] $I(p_0,t_0)\leq low$
        \end{itemize}
        where
        \[
            low=\min\{M_0(p_0)\}\cup\{\boxplus(p_0, t)\mid t\in p_0^\boxminus\}
        \]
        &
        \begin{itemize}[leftmargin=10mm]
            \item[UO1)] Remove $t_0$.
        \end{itemize} \\ \hline
        \end{tabular}
    \end{adjustbox}
    \caption{Rule O: Inhibited transition}
    \label{fig:rule_o}
\end{figure}

\begin{theorem}
    Rule~O in Figure~\ref{fig:rule_o} is correct for CTL* cardinality properties.
\end{theorem}
